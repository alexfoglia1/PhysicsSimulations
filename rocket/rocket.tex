\documentclass[10pt,a4paper]{book}
\usepackage[utf8]{inputenc}
\usepackage{amsmath}
\usepackage{amsfonts}
\usepackage{amssymb}
\usepackage{graphicx}
\usepackage[left=3.00cm, right=3.00cm, top=3.00cm, bottom=3.00cm]{geometry}
\title{Moto dei Razzi}
\date{}
\begin{document}
\section*{Moto dei Razzi}
Lo studio del moto dei razzi è un caso particolare dello studio del moto dei sistemi a massa variabile. Un razzo è un oggetto che utilizza l'espulsione ad alta velocità di una materia gassosa per poter accelerare. Solitamente questo processo avviene tramite combustione della materia gassosa.\\*
La prima equazione cardinale della dinamica afferma che, in un sistema di punti materiali, la risultante delle forze esterne è pari alla derivata rispetto al tempo della quantità di moto totale del sistema.
$$
\mathbf{P(t)} = \sum_{i} m_i \mathbf{v}_{i}(t)
$$
In un corpo rigido vale:
$$
\mathbf{v_i}(t) = \mathbf{v}(t)\; \forall i
$$
$$
\sum_{i} m_i = M\;
$$
Dove $M$ è la massa totale del corpo rigido. Quindi:
$$
\mathbf{P}(t) = M\;\mathbf{v}(t)
$$
$$
\mathbf{dP} = M\;\mathbf{a}(t)\;dt
$$
$$
\frac{\mathbf{dP}}{dt} = M\;\mathbf{a}(t)
$$
Quindi
$$
\frac{\mathbf{dP}}{dt} = \mathbf{F}^{ext}
$$
La massa totale di un razzo, al tempo $t$, è data in generale da una funzione del tempo poichè essa non è costante:
$$
M = M(t)
$$
Al tempo $t$ vale pertanto:
$$
\mathbf{P}(t) = M(t)\;\mathbf{v}(t)
$$
Mentre al tempo $t + dt$ vale:
$$
\mathbf{P}(t + dt) = [M - dm]\;\mathbf{v}(t + dt)
$$
Dove $dm$ è la quantità di massa di combustibile espulsa nell'intervallo di tempo $dt$.\\*
Questa espressione non è ancora l'espressione della quantità di moto totale del sistema, infatti manca la quantità di moto del gas espulso:
$$
\mathbf{P}(t + dt) = [M - dm]\;\mathbf{v}(t + dt) + dm\;\mathbf{v^*}(t + dt)
$$
Dove $\mathbf{v^*}(t + dt)$ è la velocità del gas espulso dal razzo all'istante $t + dt$. Osserviamo che le quantità $
\mathbf{P}(t + dt)$ e $\mathbf{v^*}(t + dt)$ sono osservate nel sistema di riferimento del laboratorio.\\*
Un osservatore che si trova nel sistema di riferimento del razzo vedrà sempre il gas uscire con una velocità $\mathbf{u}$ costante, che dipende dalla chimica di combustione del materiale gassoso.\\*
Siano $(R - O)$ il vettore posizione del razzo rispetto al sistema di riferimento dell'osservatore, e $(G - O')$ il vettore posizione del gas espulso rispetto al sistema di riferimento del razzo.\\*
Il vettore $(G - O)$ che esprime la posizione del gas espulso rispetto al sistema di riferimento del laboratorio può essere determinato attraverso la geometria dei vettori, infatti:
$$
(G - O) = (R - O) + (G - O')
$$
E di conseguenza, derivando rispetto al tempo ambedue i membri, ottengo:
$$
\mathbf{v^*} = \mathbf{v}_{razzo} + \mathbf{u}
$$
A questo punto posso riscrivere l'espressione della quantità di moto al tempo $t + dt$ come:
$$
\mathbf{P}(t + dt) = [M - dm]\;[\mathbf{v}(t) + d\mathbf{v}] + dm\;[\mathbf{u}(t + dt) + \mathbf{v}(t) + d\mathbf{v}]
$$
Dove ho adottato la semplificazione
$$
\mathbf{v}(t + dt) = \mathbf{v}(t) + d\mathbf{v}
$$
Adesso posso scrivere la variazione della quantità di moto totale del sistema come:
$$
d\mathbf{P} = \mathbf{P}(t + dt) - \mathbf{P}(t)
$$
$$
d\mathbf{P} = [M - dm]\;[\mathbf{v}(t) + d\mathbf{v}] + dm\;[\mathbf{u}(t + dt) + \mathbf{v}(t) + d\mathbf{v}] - M\mathbf{v}(t)
$$
$$
d\mathbf{P} = M\mathbf{v}(t) + M(t)d\mathbf{v} - dm\mathbf{v}(t) - dm\;d\mathbf{v} + dm\mathbf{u}(t + dt) + dm\mathbf{v}(t) + dm\mathbf{v}(t) + dm\;d\mathbf{v} - M\mathbf{v}(t)
$$
Semplificando mi rimane:
$$
d\mathbf{P} = Md\mathbf{v} + dm\mathbf{u}(t + dt)
$$
Assumendo $\mathbf{u}$ costante posso scrivere:
$$
d\mathbf{P} = Md\mathbf{v} + dm\mathbf{u}
$$
Per la prima equazione cardinale della dinamica si ha:
$$
d\mathbf{P} = \mathbf{F}^{ext}\;dt
$$
Quindi
$$
Md\mathbf{v} + dm\mathbf{u} = \mathbf{F}^{ext}\;dt
$$
$$
\Downarrow
$$
$$
M\frac{d\mathbf{v}}{dt}+ \frac{dm\mathbf{u}}{dt} = \mathbf{F}^{ext}
$$
$$
\Downarrow
$$
$$
M\frac{d\mathbf{v}}{dt} = \mathbf{F}^{ext} - \frac{dm}{dt}\mathbf{u}
$$
Osservando che $dM = - dm$ possiamo riscrivere questa espressione come:
$$
M\frac{d\mathbf{v}}{dt} = \mathbf{F}^{ext} + \frac{dM}{dt}\mathbf{u}
$$
Questa equazione è nota come equazione dei razzi.\\*
$\mathbf{F}^{ext}$ è una forza che si oppone al moto del razzo, ad esempio la forza peso. La quantità $\frac{dM}{dt} \mathbf{u}$ si chiama forza di spinta ed è responsabile dell'accelerazione del razzo e può quindi sfuggire al campo gravitazionale terrestre. Questa quantità dipende da due termini: il primo stabilisce quanta massa viene espulsa nell'intervallo infinetismo di tempo, l'altra è interamente dominata dalla chimica di combustione ed è proprio la velocità con cui il gas esce dal razzo, nel sistema di riferimento del razzo. Per avere una forza di spinta elevata dobbiamo far sì che:
\begin{itemize}
	\item Grandi quantità di gas vengono espulse nel più breve tempo possibile
	\item Il gas viene espulso a grande velocità
\end{itemize}
Supponiamo di trovarci nel caso in cui la risultante delle forze esterne è nulla, e ciò avviene quando consideriamo un razzo che ormai è lontano dalla Terra. Mettiamoci inoltre nel caso in cui la velocità di espulsione del gas abbia la stessa direzione della velocità del razzo ma verso opposto.\\*
L'equazione che ci interessa è quindi:
$$
M\frac{d\mathbf{v}}{dt} =  \frac{dM}{dt}\mathbf{u}
$$
Questa equazione differenziale può essere risolta col metodo delle variabili separate, andando a eliminare formalmente il $dt$ e facendo la proiezione dei vettori lungo la direzione della velocità del razzo:
$$
M{d\mathbf{v}} =  {dM}\mathbf{u}
$$
$$
\Downarrow
$$
$$
-\frac{1}{u}\;d{v} = \frac{dM}{M}
$$
\\*
$$
\int_{v_i}^{v_f}-\frac{1}{u}\;d{v} = \int_{m_i}^{m_f}\frac{dM}{M}
$$
\\*
$$
-\frac{1}{u}\int_{v_i}^{v_f}\;d{v} = \int_{m_i}^{m_f}\frac{dM}{M}
$$
\\*
$$
-\frac{1}{u}\int_{v_i}^{v_f}\;d{v} = \int_{m_i}^{m_f}\frac{dM}{M}
$$
\\*
$$
-\frac{1}{u}(v_f - v_i) = log(m_i) - log(m_f) = log\left(\frac{m_i}{m_f}\right)
$$
Quindi, esplicitando $v_f$:
$$
v_f = v_i + u\; log\left(\frac{m_i}{m_f}\right)
$$
Sulla base di questa equazione si può calcolare la velocità massima raggiungibile da un razzo.\\*
Il razzo può essere diviso in una parte che sono le strutture meccaniche, e una parte che è il carburante. Quindi
$$
M = M_{mec} + M_{car}
$$
La massima velocità è quella che ottengo nel momento in cui ho finito di bruciare tutto il carburante. Devo valutare l'espressione della velocità finale considerando come istante iniziale quello in cui ho tutto il carburante a disposizione, e come istante finale quello in cui non ho più carburante:
$$
v_f = v_i + u\;log\left( \frac{M_{mec} + M_{car}}{M_{mec}} \right) = v_i + u \;log \left( 1 + \frac{M_{car}}{M_{mec}}\right)
$$
Per raggiungere velocità elevate, è conveniente cercare di avere $u$ molto grande piuttosto che cercare di bruciare molto carburante, poichè la dipendenza dalla massa del carburante è logaritmica, mentre la dipendenza dalla velocità $u$ è lineare. Solitamente $u \approx 2 \frac{km}{h}$.\\*
\\*
Supponiamo di trovarci nel caso in cui il razzo si trovi sulla superficie terrestre.\\*
Il razzo viene fatto partire dalla Terra con una risultante delle forze esterne non nulla e pari alla forza peso del razzo. L'equazione che ci interessa è quindi la generica equazione dei razzi:
$$
M\frac{d\mathbf{v}}{dt} = \mathbf{F}^{ext} + \frac{dM}{dt}\mathbf{u}
$$
In cui vale, in un sistema di riferimento avente l'asse delle $y$ orientato come la verticale ascendente:
$$
\mathbf{F}^{ext} = -M\;g\;\mathbf{j}
$$
Pertanto posso proiettare l'equazione dei razzi lungo l'asse delle $y$:
$$
M\frac{dv}{dt} = -M\;g - \frac{dM}{dt}u
$$
Moltiplicando per $dt$ ambo i membri si ottiene:
$$
M\;dv = -M\;g\;dt - dM\;u
$$
$$
\Downarrow
$$
$$
dv = -g\;dt - \frac{dM}{M}\;u
$$
$$
\Downarrow
$$
$$
\int_{v_i}^{v_f}dv = -g\int_{t_0}^{t_f}dt - u\int_{m_i}^{m_f}\frac{1}{M}\;dM
$$\newpage
\noindent{}Ottenendo:
$$
(v_f - v_i) = -g(t_f - t_0) - u[log(m_f) - log(m_i)]
$$
$$\Downarrow$$
$$
(v_f - v_i) = -g(t_f - t_0) - u\log\left(\frac{m_f}{m_i}\right)
$$
Di conseguenza l'espressione della velocità finale diventa:
$$
v_f = v_i -g(t_f - t_0) + u \log\left(\frac{m_i}{m_f}\right)
$$
Occorre precisare che per arrivare a questa espressione stiamo considerando costante l'accelerazione di gravità.\\*

\end{document}